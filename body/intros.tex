% !Mode:: "TeX:UTF-8"

\chapter{引言}

\section{编写目的}
本文档分析软件需求,明确实现方向,为之后的设计、测试等环节提供参考和指导。


\section{背景}
2020年的新冠疫情使得非接触式人体测温成为常态化需求,现有的单点非接触式红外测温虽然能满足一般需求,但由于没有一个对人体温度的全面感知,存在漏检问题。
本课题提出使用MLX90640红外热像传感器的热像体温测量方案,以提高非接触测温的可靠性。

该传感器利用红外探测器和光学成像物镜接受被测目标的红外辐射能量分布图形反映到红外探测器的光敏元件上,从而获得红外热像图,这种热像图与物体表面的热分布场相对应。
为了让这种热图像表示为可视化的图像,本课题需要使用Arduino ESP32完成红外传感器的数据采集,温度到可视颜色的伪彩色转换。
由于传感器的像素较低,渲染过程中还要利用多种适当的插值算法将图像分辨率从32*24提高到512*384像素,并且图像及其轮廓需要符合人眼的生理观感。 
最终的图像通过ESP32的Wi-Fi模块发送给Android移动设备,Android设备需要完成热图像显示,关键温度数据的实时更新,图像数据保存等功能。
