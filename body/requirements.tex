% !Mode:: "TeX:UTF-8"

\chapter{需求分析}

\section{需求概述}
\begin{enumerate}
\item 设计与实现Arduino ESP32与红外传感器的数据采集方案。
\item 设计与实现ESP32 Wi-Fi与Android移动端的数据交互。
\item 设计与实现红外温度到可视颜色的伪彩色编码转换。
\item 设计与实现红外热像分辨率从32*24提高到512*384像素插值转换算法,满足图像及其轮廓符合人眼的生理观感。
\item 设计与实现Android移动端APP,功能包括热图像显示,关键温度数据的实时更新,图像数据保存等功能。
\item 图像刷新速率不小于4帧/秒。
\item Android界面有良好的用户体验。
\item 红外图像与手机摄像头的自然图像叠加。
\item 手机摄像头人脸识别。
\item 人员姓名、温度等信息记录数据库。
\end{enumerate}

\section{目标用户}
本软件可用于公共场合检测出入人员体温,记录被测人员信息,并在发现温度异常时进行报警,具备在测试体温实时报警的功能。
管理人员可以随时根据时间、地点查询测试人员总数、体温异常人数以及占比等统计信息及可视化图表,管理出入人员。



\section{需求分析}
\begin{itemize}
\item MLX90640需求
    \begin{enumerate}
      \item {数据预处理}:传感器采集到原始温度数据后,删去坏点,并将修改后的数据临时保存,将预处理后的温度数据转化为易储存的标准格式。
      \item {数据传输}:通过蓝牙和WiFI将格式化的温度数据传送给Android移动端,并接收来自Android移动端发送的控制命令。
      \item {待机与唤醒}:节省电量,保证硬件性能,在非人流量密集时段通过Android移动端发送命令,停止或开始原始数据传输。
    \end{enumerate}
\item APP需求
    \begin{enumerate}
      \item {图像处理}:在后端完成红外温度到可视颜色的伪彩色编码转换,实现红外热像分辨率从32*24提高到512*384像素插值转换算法,评价算法优劣性,并且满足图像及其轮廓符合人眼的生理观感。
      \item {信息存储分析}:在数据库中,存储被测者的基本信息和测量总人数、体温正常/异常人数,生成统计图表。
      
      \item {图像显示}:显示热图像,并实时更新关键温度数据,图像刷新速率不小于4帧/秒。
      \item {图像叠加}:红外图像可以与手机摄像头的自然图像叠加,???。
      \item {温度数据统计}:统计并保存当前捕获图像的最高温度、最低温度、平均温度。
      \item {温度异常报警}:判断被测者体温是否正常,如果异常,界面进行报警提示。
      \item {信息记录}:记录每一位被测者的基本信息(姓名,体温,住址,体温是否正常),上传到数据库。
      \item {信息查询}:查询不同地区(市)、检测时间(天)的被测者基本信息,根据体温正常和体温异常分类显示。
      \item {信息统计}:统计测量总人数、体温正常/异常人数,统计异常人数增长率、在总人数中的占比,可以根据不同地区(市)、检测时间(天)查询分类显示。
      \item {搜索潜在风险人群}:跟据体温异常者的住址,检索所有住址相同人群,显示基本信息。并在登记新人员基本信息时,若检测到相同住址,进行报警。
      \item {传感器控制}:通过蓝牙、WiFi向传感器发送命令,远程控制传感器休眠/关机,调整图像刷新率。
    \end{enumerate}
\end{itemize}
